\documentclass{beamer}
\usepackage{../styles/slide-zh}

% \usepackage{../typesetting/styles/slide-zh}
% Improved Chinese font configuration with fallbacks
% \setCJKmainfont{STSong}
% \setCJKsansfont{STKaiti}  % Better sans-serif font for Chinese
% \setCJKmonofont{STFangsong}  % Better monospace font for Chinese

% Document information
\title{幻灯片样例演示}
\subtitle{使用 slide-zh 样式}
\author{isomo}
% \institute{机构名称}
\date{\today}

\begin{document}

% Title frame
\begin{frame}
  \titlepage
\end{frame}

% Outline frame
\begin{frame}{大纲}
  \tableofcontents
\end{frame}

% Introduction
\section{介绍}
\begin{frame}{介绍}
  \begin{enumerate}
    \item 这是一个使用 \texttt{slide-zh.sty} 的演示
      \begin{itemize}
        \item 支持中文排版和格式
        \item 内置了多种实用的宏包
      \end{itemize}
  \end{enumerate}
\end{frame}

% Special boxes
\section{特殊文本框}
\begin{frame}{特殊文本框}
  \begin{block}{笔记框框}
    这是一个笔记框,用于补充说明。
  \end{block}
  \begin{alertblock}{警告框框}
    这是一个警告框,用于警告用户。
  \end{alertblock}
  \begin{exampleblock}{示例框框}
    这是一个示例框,用于展示示例。
  \end{exampleblock}
\end{frame}

% Math example
\section{数学公式}
\begin{frame}{数学公式示例}
  爱因斯坦的质能方程:
  \begin{equation}
    E = mc^2
  \end{equation}

  麦克斯韦方程组:
  \begin{align}
    \nabla \cdot \vec{E} &= \frac{\rho}{\varepsilon_0} \\
    \nabla \cdot \vec{B} &= 0 \\
    \nabla \times \vec{E} &= -\frac{\partial \vec{B}}{\partial t} \\
    \nabla \times \vec{B} &= \mu_0\vec{J} + \mu_0\varepsilon_0\frac{\partial \vec{E}}{\partial t}
  \end{align}
\end{frame}

% Tables
\section{表格}
\begin{frame}{表格示例}
  \begin{table}
    \caption{简单表格示例}
    \centering
    \begin{tabular}{l|c|r}
      \toprule
      左对齐 & 居中 & 右对齐 \\
      \midrule
      数据1 & 数据2 & 数据3 \\
      长文本 & 短文本 & 100 \cite{ctex2020manual}\\
      \bottomrule
    \end{tabular}

  \end{table}
\end{frame}

\section{动画效果}
% Example of sequential text animation
\begin{frame}{文本动画示例}
  \begin{seqpara}
    \seqsent{这是第一句话,会先显示并保持高亮。}
    \seqsent{当第二句话出现时,第一句话会变暗。}
    \seqsent{第三句话出现后,前两句都会变暗。}
    \seqsent{这样可以引导观众注意力集中在当前正在讲解的内容上。}
    \seqsent{这种呈现方式特别适合需要逐步展示论点的场景。}
  \end{seqpara}
\end{frame}

\section{参考文献}
\begin{frame}
  \frametitle{参考文献}
  \bibliographystyle{alpha}
  \bibliography{references}
\end{frame}

% Thank you slide
\begin{frame}
  \centering
  \LARGE 谢谢观看

\end{frame}

\end{document}
