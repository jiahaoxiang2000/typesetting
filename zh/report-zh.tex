\documentclass{article}
\usepackage{../styles/report-zh}

% Set document information
\title{周报 isomo (\today)}
\author{isomo}
\date{\today}

\begin{document}

\maketitle

\begin{abstract}
  本周主要完成了项目的关键模块开发与实验验证。通过设计并实现新的算法结构,成功提升了系统性能并降低了资源消耗。实验结果显示,与基准实现相比,我们的方法在处理效率上实现了20\%的显著提升,为项目后续发展奠定了坚实基础。同时,完成了技术文档的初稿撰写工作,对系统架构和核心组件进行了详细说明。
\end{abstract}

\begin{weekplan}
1) 完成第二阶段算法优化工作 2) 开始撰写实验章节 3) 准备下周的项目进度汇报演示文稿
\end{weekplan}

\section{工作进展}

本节将详细介绍本周的工作进展,包括技术实现、实验分析以及文档编写等方面。

\subsection{核心算法实现}

本周主要完成了核心算法的实现与测试,关键工作包括:

\begin{enumerate}
\item 设计了新的数据结构,优化了内存使用效率
\item 实现了并行计算模块,提升了处理速度
\item 完成了边缘情况的错误处理机制
\end{enumerate}

代码实现采用了模块化设计,便于后续维护与扩展:

\begin{figure}[h]
\centering
\includegraphics[width=0.7\textwidth]{example-image-a}
\caption{系统架构示意图}
\label{fig:system_architecture}
\end{figure}

\subsection{实验结果分析}

实验采用多组对照设计,在不同数据规模和运行环境下进行了系统测试。如图\ref{fig:performance_comparison}所示,我们的方法在各种条件下均表现优异。

\begin{figure}[h]
\centering
\begin{subfigure}{0.48\textwidth}
\centering
\includegraphics[width=\textwidth]{example-image-b}
\caption{不同数据规模下的性能对比}
\label{fig:scale_performance}
\end{subfigure}
\hfill
\begin{subfigure}{0.48\textwidth}
\centering
\includegraphics[width=\textwidth]{example-image-c}
\caption{内存占用分析}
\label{fig:memory_usage}
\end{subfigure}
\caption{系统性能分析}
\label{fig:performance_comparison}
\end{figure}

\section{遇到的问题与解决方案}

在实现过程中,我们遇到了几个关键挑战:

\begin{table}[h]
\centering
\caption{问题及解决方案总结}
\begin{tabular}{|p{0.3\textwidth}|p{0.6\textwidth}|}
\hline
\textbf{问题} & \textbf{解决方案} \\
\hline
内存溢出 & 实现了动态内存分配策略,根据实际需求调整缓冲区大小 \\
\hline
并发冲突 & 采用细粒度锁机制,减少线程等待时间 \\
\hline
性能瓶颈 & 通过算法优化和缓存策略,提高了热点路径的执行效率 \\
\hline
\end{tabular}

\label{tab:problems_solutions}
\end{table}

\section{下一步工作计划}

基于当前进展,我们计划在接下来的工作中:

\begin{enumerate}
\item 进一步优化算法性能,特别是针对大规模数据处理场景 \cite{ctex2020manual}
\item 开始第二阶段功能模块的开发 \cite{mittelbach2004latex}
\end{enumerate}

\bibliographystyle{alpha}
\bibliography{references}

\end{document}
