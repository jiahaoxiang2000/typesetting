\documentclass{article}
\usepackage{../styles/note-zh}
\usepackage{bookmark}

\title{中文笔记示例}
\author{张三}

\begin{document}

\maketitle

\section{引言}
本文档展示了如何使用 \texttt{note-zh.sty} 样式来创建中文笔记。这个样式文件提供了许多有用的功能,包括数学公式排版、参考文献引用和图表处理。

\section{数学公式示例}
使用预定义的数学宏可以简化公式的输入。例如,实数集可以用 $\R$ 表示,复数集可以用 $\C$ 表示。以下是一个方程示例:

\begin{equation}
  \int_{a}^{b} f(x) \, dx = F(b) - F(a)
\end{equation}

向量可以使用 $\vecr{v}$ 来表示,矩阵可以使用 $\mtrx{M}$ 来表示。统计中,期望值表示为 $\E[X]$。

\section{格式化功能}
本样式提供了多种格式化功能:

\begin{itemize}
  \item \todo{这是一个待办事项} 显示为红色
  \item \highlight{重要信息} 使用黄色背景高亮显示
\end{itemize}

技术名词可以使用 \term{术语} 命令来强调。

\section{引用功能演示}
引用文献是学术写作的重要部分。例如,我们可以引用高德纳的经典著作\cite{knuth1984texbook}或LaTeX指南\cite{lamport1994latex}。

\section{表格示例}
下面是一个使用 \texttt{booktabs} 包创建的表格示例:

\begin{table}[htbp]
  \centering
  \caption{数据示例表}
  \begin{tabular}{lcc}
    \toprule
    项目 & 数量 & 单价 \\
    \midrule
    苹果 & 5 & ¥2.5 \\
    香蕉 & 3 & ¥1.5 \\
    橙子 & 2 & ¥3.0 \\
    \bottomrule
  \end{tabular}
\end{table}

\section{分栏示例}
\begin{multicols}{2}
  这是分栏文本的示例。在学术论文和技术文档中,有时需要使用多栏布局来提高空间利用率。通过使用 \texttt{multicol} 包,可以轻松创建多栏布局。

  这种布局特别适合词汇表、索引或包含多个短段落的内容。注意分栏文本如何在列之间自动平衡。
\end{multicols}

\section{结论}
本文档展示了 \texttt{note-zh.sty} 样式文件的主要功能。通过使用这个样式文件,可以轻松创建具有专业排版质量的中文文档。

% 添加参考文献
\bibliographystyle{plain}
\bibliography{references}

\end{document}
